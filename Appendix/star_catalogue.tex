\documentclass[../../main.tex]{subfiles}

%-----------------------------------------------------------%
\begin{document}
\section{Star Catalogue}
\label{appendix:star_catalogue}
\thispagestyle{fancy}

%-----------------------------------------------------------%

A star catalogue, is an astronomical catalogue that lists down the stars, along with other details of these stars.
There are different star catalogues available which have been produced for different purposes over the years.
The completeness and accuracy of these catalogues are generally categorised by the weakest Limiting Magnitude and the accuracy of the positions of the stars in the catalogue. 
Limiting Magnitude is defined as the faintest apparent magnitude\footnote{Apparent magnitude is a measure of the brightness of a star or other astronomical object observed from the Earth. The magnitude scale is reverse logarithmic: the brighter an object is, the lower its magnitude. An object that is measured to be 5 magnitudes higher than another object is 100 times dimmer} of a celestial body that is detectable by a given instrument.

For 

\subsection{SKY2000 Star Catalogue}



%----------------------------END----------------------------%
\end{document}