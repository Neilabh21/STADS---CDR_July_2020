\documentclass[../../main.tex]{subfiles}

%-----------------------------------------------------------%
\begin{document}
\section{Integration}
\begin{enumerate}
    \item \textbf{Interface Control Document:} Documentation of all interface details which contains List of components, Interface Matrix, Details of Interface and Integration Sequence is done here.
    \item \textbf{Integration Sequence:} The importance of modular integration from Spacers and solid rod is known for small cubesats/modules. The assumption in this integration sequence is that the electrical interface of PCB is done directly only to the just below PCB or just above PCB.
    \begin{enumerate}
        \item Fix Bottom Panel.
        \item Interface all 4 solid rods using threads like a nut.
        \item Insert 4 Bottom spacers inside solid rod symmetrically.
        \item Fix Micro-controller PCB using solid rod and support through Bottom spacers.
        \item Insert 4 Middle spacers inside solid rod symmetrically.
        \item Position FPGA PCB using solid rod and later interface electrically through the connector with Micro-controller PCB.
        \item Insert 4 Middle spacers inside solid rod symmetrically.
        \item Position Optics PCB using solid rod and later interface electrically through the connector with FPGA PCB.
        \item Interface all 4 Top spacers with a solid rod using threads like nut and bolt.
        \item Interface Top Panel with Top spacers using screws.
        \item Interface Baffle and Side Panels.
    \end{enumerate}
    \item \textbf{Problems with Integration with Support rails:} Usually for attaching PCB with stubs on support rails, the screw head should be on the side along with PCB. To tighten a screw we need a torque wrench which requires accessibility from surrounding. As of now, we have decided to keep this model on hold because of following reasons. We might continue with it in future.
    \begin{enumerate}
        \item We need to reduce the stubs from 5 to 3 in the support rails. We have to conduct integration of the 3 stubs with as minimum movement and pre-stress as possible, and we will require a rotational motion for the integration as a solution proposed by Sanket Team for integration of top and bottom PCB.
        \item In the middle PCB, we will have to change the screw position. PCB size will become reduced hence it will not allow FPGA PCB to be designed in that constraint as FPGA itself is a big component.
    \end{enumerate}
\end{enumerate}
%----------------------------END----------------------------%
\end{document}