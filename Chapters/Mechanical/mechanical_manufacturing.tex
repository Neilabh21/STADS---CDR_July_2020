\documentclass[../../main.tex]{subfiles}

%-----------------------------------------------------------%
\begin{document}
\section{Manufacturing}
\begin{enumerate}
    \item \textbf{Fillets and chamfers:}
    \begin{enumerate}
        \item Fillets and chamfer are done to correctly model the manufactured part and also to provide info to the vendor on which edge is critical through drawing files.
        \item For manufacturing, if an edge is not critical then it must be done fillet with the radius as same as the radius of the tool(radius between 0.5mm to 1mm) since increasing tool radius saves money and time as well. 
        \item In any way, the perpendicular edge cannot be done completely perfectly, so if there is a mate at that edge then there will be interference. To solve this interference the corresponding edge on the other part must have the chamfer with the same distance as the radius of the fillet. 
    \end{enumerate}
    \item \textbf{Manufacturing Methods:}
    \begin{enumerate}
        \item Machining Process for Aluminum parts will be mostly milling. Wire EDM can also be used in some places.
        \item Drilling holes in different directions on the same component requires different fixtures to be used by the vendor. 
    \end{enumerate}
\end{enumerate}
%----------------------------END----------------------------%
\end{document}