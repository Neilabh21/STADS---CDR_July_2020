\documentclass[../../main.tex]{subfiles}

%-----------------------------------------------------------%
\begin{document}
\chapter{Introduction}
\thispagestyle{fancy}
The IIT Bombay Student Satellite Project (IITBSSP) is a landmark project by the students of IITB. Our vision is to convert IIT Bombay into a centre of excellence in Satellite and Space Technology. Pratham, the first (student) cube-satellite conceived in India was launched on September 26, 2016, onboard PSLV C-35. Its beacon signal was received on $28^{th}$ September and then again in December, at our ground station.

Currently we are developing multiple space-systems designed for CubeSats that include: Antenna Deployment System (SANKET), Star-Tracker based Attitude Determination System \emph{(STADS)}, the Great Lunar Expedition for Everyone \emph{(GLEE 2023)}, Rendezvous \& Docking \emph{(RVD)} and HAM Radio Club \& Ground Station Segments \emph{(HRC-GSS)}.

The overall aim of IITBSSP is enabling students to gain knowledge in the field of satellite and space technology, empowering the satellite team with the skill to develop the satellite through various phases of design, analysis, fabrication and testing until the flight model is launch ready. By building space-systems such as STADS, we plan to further expand our social goal of involving and empowering students from other universities in satellite building and sharing the knowledge and the skill set accumulated by the team over the years with various aspiring universities, thereby benefiting India. Satellite 101 Wiki, which is a systematic and organized compilation of the knowledge and the experience gained by the team through the journey of Pratham, aimed at helping various students and colleges across India, who wish to start off their own satellite projects is now online.


\section{Subsystems}

    \subsection{Mechanical}
    
    
    \subsection{Instrumentation}
    
    
    \subsection{Electrical}
        Electrical Subsystem is responsible for designing the PCB which includes the selection of on-board electronic components that will run the algorithm and integrate the software and hardware modules along with their interfacing. It is also involved in detecting and obtaining the centroids of stars present in an image.

    \subsection{Guidance, Navigation and Control}


\end{document}