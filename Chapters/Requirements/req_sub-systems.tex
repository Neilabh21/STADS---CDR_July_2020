\documentclass[../../main.tex]{subfiles}

%-----------------------------------------------------------%
\begin{document}
\chapter{Requirements on Subsystems}
\thispagestyle{fancy}

%-----------------------------------------------------------%
\section{Requirements on Mechanical Subsystem}
\subsection{From System}
\begin{enumerate}
        \item Size of the module should be restricted to 5cm *5cm *10cm (with Baffle).
        \item To make our system within 200 grams (w/o Baffle) 300 grams (w/ Baffle).
        \item System should be designed to sustain Launch Loads.
        \item System should provide a mechanical interfacing link to the user satellite.
        \item Maintain the temperature range of all components within specified limits.
        \item Out-gassing of the system to be avoided.
        \item System should be integrable, as a module.
    \end{enumerate}
%\subsection{From Communication and Payload Subsystem}  
\subsection{From Instrumentation Subsystem}
\begin{enumerate}
        \item Integrate the Optics PCB.
        \item Integrate the Baffle.
        \item Seperation is required between Baffle and Optics PCB
    \end{enumerate} 
\subsection{From Electrical Subsystem}
\begin{enumerate}
        \item Integrate the FPGA PCB.
        \item Integrate the Micro-controller PCB.
        \item Separation is required between all pairs of PCB for positioning of connectors
    \end{enumerate}
\subsection {From Testing Considerations} 
%----------------------------END----------------------------%


%----------------------------------------------------------%
\newpage
\section{Requirements on Instrumentation Subsystem}
\subsection{From System}
\begin{itemize}
    \item Selection of sensors and optics system should cater to the required system accuracy of 0.01$^{\circ}$.
    \item Should be able to capture workable images at a max slew rate of 2$^{\circ}$/sec such that effective image processing can be done.
    \item To select the lens, imaging sensor and baffle which are space-grade.
    \item Exposure time should be optimized.
    \item Field of view should be decided such that desired sky coverage is achieved.
    \item Baffle sun rejection (exclusion) angle should be greater than 40$^{\circ}$.
\end{itemize}
\subsection{From Electrical Subsystem}
\begin{itemize}
    \item The imaging sensor selected should operate below power requirement (to be decided after power budget).
    
\end{itemize}
\subsection{From GNC Subsystem}
\begin{itemize}
    \item Optical system should be capable of observing stars of the specified maximum magnitude of 6.
    \item At least 4 detectable stars of specified magnitude should be visible in the image.
\end{itemize}

\subsection{From Mechanical Subsystem}
\begin{itemize}
    \item Lens, imaging sensor, and baffle should be within a specific volume range so as to maintain the system volume of : \\$5cm\times5cm\times10cm$ %(w/o baffle) \\$5cm\times5cm\times18cm$ (w baffle)
    \item Lens, imaging sensor, and baffle should weigh under a specific weight so as to maintain the system weight of:\\200 gms (w/o baffle)\\300 gms (w baffle)
    \item To select lens, sensor, and baffle that can tolerate the specified temperature range of -40$^\circ$C to +50$^\circ$C. 
\end{itemize}
%------------------------Lens, imaging sensor, and baffle should weigh under the specified weight.----------------------------------%


%-----------------------------------------------------------%
\newpage
\section{Requirements on GNC}
\subsection{From System}
\begin{itemize}
    \item Estimator must be capable of giving attitude with required accuracy with the obtained data
    \item Algorithms should be selected which are less computationally expensive
    \item The pre-processed star catalogue used onboard should consist of data from stars of specfied upper magnitude.
\end{itemize}
\subsection{From Testing Considerations}
%----------------------------END----------------------------%


%-----------------------------------------------------------%
% \newpage
% \section{Requirements on Communication subsystem}
% \subsection{From Mission}
% \subsection{From Payload}
% \subsection{From Testing Consideration}
% \subsection{From Electrical Subsystem}
% \subsection{From Mechanical Subsystem}
%----------------------------END----------------------------%


%-----------------------------------------------------------%
\newpage
\section{Requirements on Electrical Subsystem}
\subsection{From System}
\begin{itemize}
    \item System shall operate with power requirement less than 1.5 W.
    \item The algorithms should be able to process images and give star coordinates on image plane within reasonable bounds.
    \item The algorithms should be able to process images and give attitude within reasonable bounds at slew rates up to 2$^\circ$/sec.
    \item To select components which are space-grade.
    \item Should provide a power and communication link between STADS system and user.
    \item Overall electrical interface and components should give output in a cycle of specified time.
\end{itemize}
\subsection{From Payload and Communication Subsystems}
\begin{itemize}
    \item 
\end{itemize}
\subsection{From Testing Considerations}
\begin{itemize}
    \item 
\end{itemize}
\subsection{From GNC}
\begin{itemize}
    \item Suitable micro controller to be selected that can run the algorithms in the stipulated time.
    
\end{itemize}
\subsection{From Mechanical Subsystem}
 \begin{enumerate}
    \item Accommodate all the components on FPGA and Microcontroller PCBs of size less than 45mm x 45 mm x 1.5mm.
    \item Electrical components, and the PCBs should weigh under the specified weight.
    \item To select components that can tolerate the specified temperature range of -40$^\circ$C to +50$^\circ$C.
   \end{enumerate}
%----------------------------END----------------------------%
\end{document}