\documentclass[../../main.tex]{subfiles}

%-----------------------------------------------------------%
\begin{document}
\section{PCB}
\thispagestyle{fancy}

The FPGA selected for our STADS system is Xilinx XC7Z010. This FPGA is available in 2 packages CLG225 (225 pin BGA) and CLG400 (400 pin BGA). Initially we selected CLG225 since we had never worked on a BGA before and the complexity of routing increases as the number of pins increase. However, we had to shift to CLG400 package because the CLG225 package doesn't support boot via $\mu$SD card.  As FPGA has volatile memory we have planned to program it via a $\mu$SD card whenever it boots up.

The 400 pins of this FPGA are segregated into different banks such as bank 34, 35, 501, 502 and 0. Each bank of pins has it's own power supply and is capable of operating at various voltages as required. Some banks, such as 501 for instance, however need to be operated at a certain fixed voltage, 1.8V in this case. Bank 34 and 35 contain the GPIO pins for both differential as well as normal signals. Banks 501 and 502 connect the PS to the peripherals such as USB. Bank 0 contains the configuration pins.

The features of this FPGA are
\begin{itemize}
    \item {QSPI flash memory }
    \item {PMOD interface}
    \item {Gigabit Ethernet port}
    \item {USB OTG}
    \item {USB-UART interface}
    \item {Micro SD card}
    \item {DDR RAM}
    \item {JTAG}
    \item {Different I/O banks}
\end{itemize}

Since FPGA's are available in a BGA(Ball grid array) package, they need multiple layer PCB to be able to access the pins near the centre. We shall be using a 6 layer PCB setup comprising of 4 signal planes, 1 GND plane and 1 VCC plane. When working on a multi layer board, there are several constraints that need to be taken care of. The constraints on the stack up of the PCB state that each signal layer should have at least one reference plane(GND or VCC) adjacent to it. These constraints are imposed in order to decrease EMI (Electromagnetic Interference) caused due to the different kinds of i/o pins and integrated chips. To maintain signal integrity and avoid cross-talk, it is also suggested that routing on adjacent layers should be in perpendicular directions.

Another set of constraints are those pertaining to vias. In multi layer boards, there are different type of vias.
\begin{itemize}
    \item \textbf{Through hole via} - one which passes through the entire board and connects the top layer to the bottom layer.
    \item \textbf{Blind via} - one which connects either the top or the bottom layer to one of the inner layers. As a result, the via can only be seen from one side, hence the name 'Blind via'.
    \item \textbf{Buried via} - one which connects two internal layers. So this cannot be seen from either side and is sandwiched between the top and bottom layers, hence the name 'Buried via'.
\end{itemize}

We also have constraints from manufacturing side that need to be taken into account such as the minimium trace width that can be laid, minimum gap between the traces, minimum drill size, the number of layers that the manufacture can make and the different types of vias that can be placed by the manufacturer. All the constraints mentioned above can be implemented by changing certain variables in the EAGLE DRC(Design Rule Check)

While sending objects in space we should bear in mind the harsh conditions in space like the extreme temperature cycles, radiations etc. This requires proper calculations to be done during PCB designing and selecting proper components such as military grade or space grade. Since our FPGA will be operated at high frequencies, it is important to perform proper thermal and impedance calculations. Controlled impedance is important for signal integrity: it is the propagation of signals without distortion.

Being a fairly complex system, the FPGA provides for differential signalling as well on most of the GPIO pins. A differential signal is basically a signal that travels on a differential pair of routed wires. The differential signal consists of two signals wherein the second signal is simply a negation of the first (high becomes low and vice versa). This is done in order to reduce errors due to loss or corruption of data travelling through signal routes. This leads to an important design constraint. The length of the two signals forming a differential pair must be the same. If the lengths differ more than a certain tolerance, it may lead to the system not working properly. The two signals will arrive at different times and if the time difference is greater than one clock cycle, it will cause errors.

FPGA requires various voltage levels to operate properly and they should be powered up in a particular sequence in order to initiate a successful boot. To achieve this, we use linear voltage regulators. In our case we need 1V, 1.5V, 1.8V and 3.3V voltage rails. We are using TLV62130RGT voltage regulators. These voltage regulators have enable pins and power good pins. The voltage regulators begin functioning when a high signal is provided to the enable pins and the power good pins give a high output when the voltage has stabilised around the required level. To acheive the sequencing of the voltage levels, we cascade the various regulators in such a way that the power good of one regulator is connected to enable pin of the next regulator. 

We use multiple capacitors instead of a single large capacitor of the same equivalent capacitance because it lowers the ESR, which is the internal resistance of the capacitor limiting the rate at which it can supply it's stored energy back to the circuit. Using a single capacitor will not filter supply line noise over the entire frequency range needed, so a mix of capacitor values and types are used in parallel whereby a much wider range of noise frequencies can be filtered. Also using multiple capacitors increases the total surface area, hence improving heat dissipation and thus extending the life of the capacitors.
%----------------------------END----------------------------%
\end{document}