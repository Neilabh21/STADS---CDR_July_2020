\section{Feature Extraction}
\thispagestyle{fancy}

Feature Extraction is the process of obtaining the centroids of all detectable stars, with maximum accuracy. It takes in the image obtained by the image sensor and outputs the location of the corresponding centroid of various stars in image plane coordinate.

We are working on three different algorithms for Feature Extraction, and selection between them will be done on the basis of performance, time-wise and accuracy-wise (along with availability of resources in case of the sequential algorithms).
\begin{itemize}
    \item Sequential Algorithms:\\
    \ These will be implemented on the FPGA, provided that enough resources are left over after interfacing is implemented. We have two sequential algorithms that we plan to test on the FPGA:
    \begin{itemize}
        \item The pixel-by-pixel tagging algorithm:\\
        \ This algorithm goes through all the pixels in the image and assigns a tag to each pixel. Tags that get connected at a pixel are merged at the end of tagging. This is a modification of one of the classical algorithms used for image labelling. An example of how this labelling algorithm works is presented at \href{https://aishack.in/tutorials/labelling-connected-components-example/}{this webpage}.
        \item The line-by-line algorithm:\\
        \ This algorithm compares the starting and ending points of continuous ranges of pixels in every row and assigns tags to those ranges. Tags that get connected to the same range are merged at the end of comparisons for each row. The algorithm to be implemented is presented in \cite{fe_blob_detection}.
    \end{itemize}
    \item Recursive Algorithms:\\
    We are working on a recursive centroiding algorithm called the region growth algorithm. Recursive algorithms can not be implemented on the FPGA as HLS can not synthesize recursive functions. Therefore, this algorithm will run on the microcontroller.
\end{itemize}

All three of these work on the basis of thresholding. The pixels with intensity above a pre-set threshold are said to be bright, and the algorithms define stars as 4-connected regions of bright pixels. 

Two pixels are said to be 4-connected if they share an edge. Thus, for a given pixel, the pixels to the immediate "north", left, right and "south" are said to be 4-connected to it and vice-versa. 4-connected regions can be defined quite easily using this definition.

There are lower and upper bounds on the number of pixels in each star to account for anomalous readings and large celestial bodies that are not stars and hence can't be matched by the star matching algorithms.

\large{\textbf{Finding the centroid for each star:}}\\
\normalsize
There are two centroiding methods which are commonly employed by star trackers:
Center of gravity and Gaussian curve fitting.

\textbf{Gaussian Curve Fitting:}
The pattern of light caused by a single star on the image plane can be approximated as a 2D Gaussian function. Gaussian curve fitting attempts to fit a Gaussian function to the pattern of light caused by each star. Once this is achieved, the centroid of the star can be found by calculating the location of the fitted Gaussian function's peak. This is the more accurate of the two methods.

\textbf{Center of Gravity:}
Compared to Gaussian curve fitting, a Center of gravity equation is less accurate, but is much less processor-intensive and so we face a trade-off between the time taken by the process and its accuracy and which process is to be employed must be decided based upon the system's requirements.

According to the center of gravity approach, the centroid of each each star is defined as
\begin{align}
    C_{x} & := \frac{\sum C_{i_{x}}Intensity_{i}}{\sum Intensity_{i}}\\
    C_{y} & := \frac{\sum C_{i_{y}}Intensity_{i}}{\sum Intensity_{i}}
\end{align}

In the Gaussian fitting algorithm, an initial centroid is found out using the Center of gravity method, then the pixels are assigned an additional weight which is given by Gaussian probability distribution function.

We have decided to go with the the center of gravity approach because of the simplicity and the computational cost of implementing the Gaussian Curve Fitting algorithm.

Currently, the pixel-by-pixel algorithm and the region growth algorithm have been coded and tested on both MATLAB and C++. These testing results have been compiled and presented in section \ref{sec:Electrical_test}.

We now present the details of the algorithms.

\documentclass[../../main.tex]{subfiles}

%-----------------------------------------------------------%
\begin{document}
\subsection{The Pixel-By-Pixel Tagging Algorithm} \label{subsec:fe_pixel}

This algorithm is a modified version of the image labelling algorithm presented in \cite{imglabelseq}. The input is the image captured by the image sensor, and the output is an array containing the centroids of 4-connected regions of bright pixels in the input image in the frame centered at the center of the image and scaled to the size of the pixels (rather than in units of pixel lengths), along with Star IDs that are set as an integer between 1 and N, with N being the number of stars detected by the algorithm. These are allotted in the order of the position of the stars in the output array.

We now describe the algorithm:
\begin{enumerate}
    \item A row of zeros is added to the north of the input image and a column of zeros is added to both the left and right of the input image. The algorithm iterates over the input image from the top left to the bottom right of the image (excluding the added zero elements)
    \item Whenever the algorithm encounters a bright pixel, it checks the pixel to the north and to the left. We store $ \sum C_{i_{x}}Intensity_{i}, \sum C_{i_{y}}Intensity_{i}, \sum Intensity_{i} $, the number of pixels and the flag for each tag\footnote{Note that a bright pixel that has already been inspected by the algorithm will \emph{always} have a tag.}.
    
    The following cases are encountered while checking the neighbouring pixels:
    \begin{enumerate}
        \item If both of them are bright, it checks their tags
        \begin{enumerate}
            \item If both of the tags are the same, the data of the pixel is added to the data of the corresponding tag
            \item If both of them are different, it sets flags to note the equivalence relationship, since they get connected at a pixel. We also have a table to list the tags associated with every flag. It checks the flags associated with both flags.
            \begin{itemize}
                \item If both the tags have the same flag, nothing needs to be done
                \item If only one of the tags has a flag, it flags the other tag with that flag. The newly flagged tag is added to the table in the corresponding row
                \item If both the tags have different flags, it merges the two rows in the table to the one with the smaller flag and updates the flag of all tags in the row
                \item If none of them have flags, it creates a new flag and flag the tags, and adds them to the new row in the table
            \end{itemize}
        Regardless of what the flags are, it adds the data of the pixel to the data of the tag of the pixel to the north and tags it with the same.
        \end{enumerate}
        \item If only one of them is bright, the pixel is tagged with the corresponding tag and the data is added to the data of that tag
        \item If none of them are bright, it checks the pixel to the right and the one to the right of the one to the north\footnote{This is done to reduce the unnecessary creation of new tags and the resulting conflicts between tags that will eventually get connected at some pixel. The testing done so far has suggested that this is fairly effective in cutting down on the use of flags and the resulting merge.}.
            \begin{itemize}
                \item If both of those are bright, the pixel is tagged with the tag of the pixel to the right of the one to the north. 
                \item If not, the pixel is assigned a new tag and the data is added to the corresponding tag's data
            \end{itemize}
    \end{enumerate}
    \item After each pixel is tagged, it no longer needs the input image, all the data we need is stored with the data of each tag. It goes through the flags for each tag and combines the data of the flagged ones. 
    \item It then goes through all the data corresponding to the merged tags and the unflagged tags and filters out the ones that have too few or too many pixels and finds the centroids for the remaining regions.
    \item The Star IDs are then added to the array and this, along with the number of regions, is given as the final output.
\end{enumerate}
The flowchart for this algorithm is shown in Figure \ref{FC:pbp_centroiding}.
%----------------------------END----------------------------%
\end{document}