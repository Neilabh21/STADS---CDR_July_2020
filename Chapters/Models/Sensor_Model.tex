\documentclass[../../main.tex]{subfiles}

%-----------------------------------------------------------%

\begin{document}

% ===================== %
% Star Image Simulation %
% ===================== %

\newpage
\section{\SIS \label{sec:SISM}}


% ---------- %
% Background %
% ---------- %

\subsection{Background} % BG
There are mainly three approaches for testing in star identification algorithm research: digital simulation, hardware-in-the-loop simulation, and field test of star observation. These three approaches correspond to three different stages in designing a star sensor and a star identification algorithm.

\subsubsection{Digital Simulations}
At the early design stage, preliminary performance evaluation of the algorithm is done using digital simulation to determine appropriate identification parameters. Digital simulation is computer-based and is involved in the whole process of star image simulation, star image processing, and star identification. 

\subsubsection{Hardware In-Loop Simulation}
After the design of the star sensor finishes, star identification algorithms can be verified using the method of hardware-in-the-loop simulation. Through hardware-in-the-loop simulation, star field simulator (SFS) generates star images. Then the imaging, processing, and identification of those generated star images are done by star tracker. 

\subsubsection{Field Testing}
Field tests of star observation are done during the night. Star images are photographed and then identified by the star sensor method in order to further verify the star identification algorithms. 

\subsubsection{Star Image Simulation}
As for star identification algorithms, star image simulation is the fundamental work of the research. That is, when star sensor’s attitude or boresight pointing is given, those star images photographed by star sensor can be simulated.

% ---------- %
% Motivation %
% ---------- %

\subsection{Motivation} % Why
When designing an optical high precision pointing system many factors contribute to the pointing performance. It may be necessary to consider several system architectures and trade offs amongst several hardware options. Another choice is in terms of the Feature Extraction, Star Matching (Lost in Space and Tracking Mode, both) and Estimation algorithms - it may be necessary to verify that the current set of algorithms can achieve the desired accuracy. However, the conventional approach of hardware testing is impossible for a space-based system. 

One cost-effective method to perform hardware trade studies and to analyze performance is to create a computer simulation of the optical hardware. There are several noise sources from the selected detector and theoretical optical limitations which affect the system performance. The simulation must be \emph{robust} to capture the \emph{dominant noise sources} which impact the downstream pointing precision. 

Therefore, the goal for \SIS (SIS) is -
\begin{displayquote}
    \emph{``To develop a robust model of the lens system and image sensor that can generate simulated star field images which include optical aberrations and sensor noises."}
\end{displayquote}
The images can be analyzed with simulations to characterise the centroid error.

% ------------ %
% Introduction %
% ------------ %

\subsection{Introduction} % What

As mentioned earlier, the goal for the \SIS is to develop a robust model of the lens system and image sensor that can generate simulated star field images which include optical aberrations and sensor noises. 

Thus, given the sensor's attitude or boresight pointing and angular velocities, this robust \SISM (SISM) should be able to simulate the output of the CMOS Image Sensor. 

% ---------- %
% Iterations %
% ---------- %

\subsection{Iterations of the \SISM}
The Sensor Model has been developed in multiple iterations. The first iteration was in Python. When the team moved to Matlab in January 2020, The codes in Python were ported to MATLAB, without any major changes to the code. Hence, this is again the Version 1 (MATLAB). However, at this point, we realised that the structuring of the code was not compliant to the team's OOP Practices. Hence, the Version 2 (MATLAB) was created. Eventually, we realised that Version 2 can't be modified to allow for slew rate, and hence, a new version is needed. This is called the Version 3, and is to incorporate Slew Model, Lens Model as well as use the Dynamics Model. This is currently under development.

\subsubsection{Version 1 (Python)}
The first iteration of the Sensor Model was written in Python. This included a very basic model of the image formation, with no noises due to the sensor, or the aberrations due to the lens. The star vectors in the Lens Frame are calculated, back projected on to the Sensor Frame (2D), and a Gaussian Point Spread Function is added to the image. The codes for this are available at 
\href{https://github.com/Neilabh21/STADS_v1_Python/tree/master/Sensor_Modelling}{SISM - Version 1 - Python}

% Put flowchart here

\subsubsection{Version 1 (MATLAB)}
This was a port of Version - 1 (Python) to MATLAB. Structurally, this was the same. The only change is in terms of the language.

\subsubsection{Version 2 (MATLAB) - Current Model}
This is the latest stable version for the Sensor Model. It is hosted at \href{https://github.com/Neilabh21/STADS_v2_MATLAB/tree/master/Sensor_Model}{SISM - Version 2 - MATLAB}. The details to this model are given in the Section \ref{subsection:v2}.


\subsubsection{Version 3 (MATLAB) - In Development}
The Version 2 is complete in itself, and has been used to generate preliminary images for the first Model In-Loop Simulation. However, the way it has been structured, it can't be modified to include the Lens Model, the Slew Model, the Ephemeris Model. Also, with the Version 2, the Dynamics Block has not yet been implemented. So, we have arrived at a new structure for the Version 3 of \SISM, which is modular and can accommodate all the deficiencies of the Version 2. The details to this model are given in the Section \ref{subsection:v3}.

% --------- %
% Version 2 %
% --------- %

\subsection{Detailed Model - Version 2 \label{subsection:v2}}

\subfile{Chapters/Models/Sensor_Model_v2}

% --------- %
% Version 3 %
% --------- %

\subsection{Detailed Model - Version 3 \label{subsection:v3}}

\subfile{Chapters/Models/Sensor_Model_v3}


% \subsection{Softwares Used}
% Initially, Python was being used 



\subsection{Future Tasks}
The future task is to implement the Version 3 of the \SISM.

% \printbibliography
%----------------------------END----------------------------%
\end{document}